\documentclass{article}
\usepackage{revy}
\usepackage[utf8]{inputenc}
\usepackage[T1]{fontenc}
\usepackage[danish]{babel}

\revyname{DIKUrevy}
\revyyear{2024}
\version{1.0}
\eta{2:30}                      % ÆNDR HÉR. 2:30 i v1.0.
\status{4--6/7}                   % ÆNDR HÉR
% 1 = Der er en idé.
% 2 = Der er skrevet store dele, men mangler stadig linjer.
% 3 = Skrevet færdig, men trænger til en del opstramning.
% 4 = Færdigskrevet og god nok til at komme på scenen, man kan stadig forbedres.
% 5 = Alt fungerer godt, selv hvis vi ikke tweaker mere overhovedet.
% 6 = Grundigt poleret: Alle linjer fungerer godt. Intet overflødigt. Tempoet er godt, og slutningen er stærk.
% 7 = Super velpoleret. Nærmest umuligt at forbedre en eneste linje.

\title{DIKU Sommeliers}                   % ÆNDR HÉR
\author{Anna Liv}

% FILNAVN ER akt-rækkefølge-navn LOWERCASE UDEN SYMBOLER
% Se https://github.com/dikurevy/Public-Archive for inspiration

\begin{document}                
\maketitle
\section*{Noter}                % ÆNDR HEREFTER
``(rug)by'' kan stå stærkere i en overflod af gode ordspil. Bedre at slutte med hvordan man kommer UD af Vin. Bør udføres med accent.
Kombinér med den anden ordspilssketch [Haskall du ha' bank?]
% 0: Jeg er nærsynet, så jeg kan ikke C#
% 1: Du lyder sjovt, har du en LISP-dialekt? Dér kan man bare C
% Kor: Plus plus
% 0: Har du overvejet, at få det lavet? Det er måske ikke det værste at gøre, sådan Objective C't
% 1: Rent funktionelt, ville det nok være at foretrække.
% 0: Bør vi ikke stoppe med de her dumme ordspil?
% 1: Ha!skall' du ha' bank?
for at det ikke er for mange vindruordspil; eller læg dem umiddelbart i forlængelse af hinanden, hvor sketch \#1 afbrydes med argumentet ``for mange ordspil''.
Mega grinern. Kan godt være længere, måske med forelæserpuns.
Flere puns, sæt sammen med anden punsketch?
Evt. slut med ``Ej, de har jo skiftet vores VIN til PEPSI (e)MAX!''

Den kunne godt have noget `pagne, når nu vi skal til at slutte.

Gardiner ordspillet skal have pause for at blive fanget (det er en Windows joke).

Der kan være en ordspilsklokke der ringer for hvert ordspil (enten AV eller band), som set-up til ding-ding-ding til sidst.

Man kan evt. sige ``velkommen til Kantinen'' i starten, og så slutte med ``nå, jeg er godt nok blevet træt, jeg må hellere lægge mig/gå ovenpå og sove'', fordi næste sang er ``ingen sover i Kantinen.'' Morten skal danse i dén sketch, så kan evt. sige at han ikke kan ``exit Vin''?

% Andre gode ordspil:
% Klaser (klasser vs. vinklaser)
% Druerspil (Drue-ordspil, druk-spil)
% Syndigt, som da der kom en Python/Boa i edens have


\begin{roles}
\role{O}[Morten] Otto Vin Bismag-k. Tung accent.
\role{S}[Anna Liv] Sommeli-Jørgen
\role{X}[Sia] Instruktør
\end{roles}

\begin{props}
    \prop{Vin} Ingen væske på scenen; medmindre man laver KemiRevy, og hælder vand på scenen... lige foran Sia.
    \prop{Glas (3)}
    \prop{Bord}
    \prop{Spons} KU Wine Society har tilbudt rekvisitter ad libitum mod at deres skilt kommer på scenen og navn bliver nævnt.
\end{props}

\newpage%
\begin{sketch}

% mangler introduktioner af navne

\scene{Lys op. O og S står på scenen ved et bord med $n$ glas vin.}

\says{O} Godaften... Mit navn er Otto Vin Bismag-k og dette er min kollega 

\says{S} Sommeli-Jørgen

\says{O} og velkommen til \emph{Druer med DIKU}, det eneste vinsmagningsprogram \emph{for} dataloger, \emph{af} dataloger! 

\says{S} Ja, og i aften har vi noget \emph{helt} særlig til jer, nemlig en gætteleg! Her har vi nogle glas vin, og så skal Otto og jeg gætte, hvilken det er, kun ud fra smagen. Lad os starte med den første her.

\scene{O og S smager på den 0. vin.}

% Sacre-fault [segfault]? Mon dieuf?
\says{O} Uha! Sikke et \textbf{PoP} af smag! Og virkelig \textbf{F\#} i sin afrunding.

\says{S} Jaa, og kroppen er godt nok \textbf{SOLID}.

\says{O} Nå... skal vi smage den næste?

% Rigtig god mulighed for at referere tilbage til z-plus sketchen!
% Hvis S rækker til O, kan O evt. svare "million tak"
\scene{O og S smager på den 1. vin.}

\says{O} \textbf{Monad}r! \textbf{GPU}-d med den!

\says{S} Ej, hvad mener du? Jeg \textbf{SML}sker den her smag!

\says{O} Njah, jeg tror den \textbf{Haskell}'t. Har den gardiner?

\says{S} Nej, jeg tror, den er \textbf{\textsc{unix}}-baseret... En meget \textbf{Apple}$\cdot$erende smag.

\says{S} Den er \textbf{objektiv}t set i en \textbf{klasse} for sig selv. 

\says{O}[smasker] Hmm... eftersmagen \textbf{Erlang}.

\says{O} Man kan virkelig smage fadets \textbf{barc}. % barc's logo på AV

\says{S} Lad os gå videre til den sidste.

\scene{O og S smager på den 2. vin.}

\says{S} Mmm ja, den \textbf{RAM}mer lige i smagsløgene.

\says{O} Ja, hva'? \textbf{Script} det lige ned.

\says{S} Det er lige sådan en, man gerne vil have til \textbf{Jul$\cdot$ia}.

%\says{O} Ja, den er dejlig \textbf{Ru}g\textbf{by}.
\says{O} Ja, og den har en flot \textbf{Ruby} farve.

\scene{O og S træder væk fra vinbordet.}

\says{O} Nå, skal vi gætte, hvad der er i glassene?

\says{S} Ja, det skal vi, og jeg tror, vi tænker det samme, Otto...

\says{O} Det tror jeg også, Jørgen...

%\says{O,S} Alle glassene er fyldt med Coca-Cola!
\says{O,S} De har skriftet vores \textbf{VIM} til pepsi \textbf{eMACS}. 

\scene{AV kører en ``ding ding ding korrekt''-agtig lyd. Det bliver somehow revealed at det hele er cola, evt. et tæppe som falder ned fra foran bordet som revealer colaspanden. Generel jubel følger.}

\says{O} \textbf{Futhark} for denne gang, og følg med i næste afsnit, hvor vi lærer at kode i \textbf{Vin}.

\says{O} Jeg er blevet træt af alt denne vin. Jeg tror, jeg går op og ligger mig i kantinen. 

\says{S} Det forstår man godt. Du drikker også som et \textbf{hulkort} i jorden.

\says{S} Jeg tror faktisk, jeg skal have mere. Hvem vil med på Caféen?


\end{sketch}

\end{document}

Hvad caller man klasen af langtrukne vittigheder der får folk til at vin af grin? Druordspil.