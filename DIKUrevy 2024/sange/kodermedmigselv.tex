\documentclass{article}
\usepackage{revy}
\usepackage[utf8]{inputenc}
\usepackage[T1]{fontenc}
\usepackage[danish]{babel}

\revyname{DIKUrevy}
\revyyear{2024}
\version{1.0}
%\eta{0:00}                      % ÆNDR HÉR           
% 1 = Der er en idé.
% 2 = Der er skrevet store dele, men mangler stadig linjer.
% 3 = Skrevet færdig, men trænger til en del opstramning.
% 4 = Færdigskrevet og god nok til at komme på scenen, man kan stadig forbedres.
% 5 = Alt fungerer godt, selv hvis vi ikke tweaker mere overhovedet.
% 6 = Grundigt poleret: Alle linjer fungerer godt. Intet overflødigt. Tempoet er godt, og slutningen er stærk.
% 7 = Super velpoleret. Nærmest umuligt at forbedre en eneste linje.

\title{Jeg koder med med mig selv}                   % ÆNDR HÉR
\author{Bertil, Rasmus}  % ÆNDR HÉR
\melody{Dancing with myself 2001 remaster af Billy Idol}     % ÆNDR HÉR

% FILNAVN ER akt-rækkefølge-navn LOWERCASE UDEN SYMBOLER
% Se https://github.com/dikurevy/Public-Archive for inspiration

\begin{document}                
\maketitle
\section*{Noter}                % ÆNDR HEREFTER
Gør gerne mere NASTAYYY, men også gerne intelligent i ordspillene - sidste vers med at dreje mis er måske liiiiidt sådan…\\
Kostume: Billy Idol i 80er læder PLEASE\\
Karakterer: Sanger, oh-kor/dansere - dj, folk der boller til organisk kemi


\begin{roles}
\role{S}[Anna Liv] Sanger
\role{K1}[Kristoffer] Korsanger 1
\role{K2}[Johnson] Korsanger 2 
\role{X}[Sia] Instruktør
\end{roles}

\begin{props}
    \prop{Rekvisit} 
\end{props}

\newpage

\begin{song}
[Vers 1]
\sings{S}%
Ud’ på flooret ned’ på C?
Eller på HCØ måske?
Kan du se jeg bli’r spændt op, når min 
macbook bli’r tændt op
Jeg koder med mig selv

Når jeg er til DIKUBar
Er det med min Emacs klar
Jeg får stillet min sult, her ved DJ’ens pult
Hvor jeg koder med mig selv \act{oh-oh}

[Omkvæd]
\sings{S}
koder med mig selv \act{oh-oh}
koder med mig selv \act{oh-oh}
Får datamaten igang, så jeg kan stille min træng \act{oh-oh}
Ved at kode med mig selv \act{oh-oh-ohoh}

[Vers 2]
Store O er ren forkælning
Så jeg kom til forelæsning
Jeg hev bæstet frem, jeg sku’ kode slam
Og jeg koded med mig selv

jeg tænder min CUMputer
Siger “baby, lad mig brug’ dig”
Stikker kablet ind, så’ vi igang igen

Og jeg koder med mig selv \act{oh-oh} 

[Omkvæd]
Koder med mig selv \act{oh-oh} 
Koder med mig selv \act{oh-oh} 
Til organisk kemi sker der ting jeg kan li’ \act{oh-oh} 
Så jeg koder med mig selv \act{oh-oh} 

[Kontrast]
oh-oh-oh
oh-oh-oh
oh-oh-oh
oh-oh-oh AUUUUU

\act{kod! kod!}
\act{kod! kod!}
\act{kod! kod!}
\act{kod! kod!}
\act{kod! kod!}
\act{kod! kod!}
\act{kod! kod!}

[Vers 3]
\sings{S}
Der så mange der gern’ vil kode
For jeg er god til at bruge hovedet
Jeg gider ikk’ snak, jeg vil ha’ no’d Piq
Men jeg koder med mig selv

Om det er dag eller nat
Så har jeg “an itch to Scratch”
Jeg vil dreje mis, og jeg vil kode F\# \scene [udtalt fis (musikjoke)]
Men jeg koder med mig selv \act{oh-oh}

[Omkvæd]
\sings{S}
Koder med mig selv \act{oh-oh}
Koder med mig selv \act{oh-oh}
Hvis jeg var alen’, vil jeg kode på scen’n
Hvis jeg var alen’, vil jeg kode på scen’n
Hvis I var alen’, vil I kode på scen’n

[Outro]
\sings{S}
Oh-oh-ohoh
Oh-oh-ohoh
Oh-oh-ohoh
Oh-oh-ohoh
Oh-oh-ohoh \act{koder med migselv}
Oh-oh-ohoh \act{koder med migselv}
Oh-oh-ohoh \act{koder med migselv}
Oh-oh-ohoh \act{koder med migselv}r

\end{song}

% Du vil måske opdele i vers, bro (pre-chorus), omkvæd, og kontraststykke (c-stykke), men det er ikke et krav.
% Det vigtige er, at dem, der skal opføre det printede manus, kan finde den information, de søger.
\end{document}